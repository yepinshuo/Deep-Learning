
% Default to the notebook output style

    


% Inherit from the specified cell style.




    
\documentclass[11pt]{article}

    
    
    \usepackage[T1]{fontenc}
    % Nicer default font (+ math font) than Computer Modern for most use cases
    \usepackage{mathpazo}

    % Basic figure setup, for now with no caption control since it's done
    % automatically by Pandoc (which extracts ![](path) syntax from Markdown).
    \usepackage{graphicx}
    % We will generate all images so they have a width \maxwidth. This means
    % that they will get their normal width if they fit onto the page, but
    % are scaled down if they would overflow the margins.
    \makeatletter
    \def\maxwidth{\ifdim\Gin@nat@width>\linewidth\linewidth
    \else\Gin@nat@width\fi}
    \makeatother
    \let\Oldincludegraphics\includegraphics
    % Set max figure width to be 80% of text width, for now hardcoded.
    \renewcommand{\includegraphics}[1]{\Oldincludegraphics[width=.8\maxwidth]{#1}}
    % Ensure that by default, figures have no caption (until we provide a
    % proper Figure object with a Caption API and a way to capture that
    % in the conversion process - todo).
    \usepackage{caption}
    \DeclareCaptionLabelFormat{nolabel}{}
    \captionsetup{labelformat=nolabel}

    \usepackage{adjustbox} % Used to constrain images to a maximum size 
    \usepackage{xcolor} % Allow colors to be defined
    \usepackage{enumerate} % Needed for markdown enumerations to work
    \usepackage{geometry} % Used to adjust the document margins
    \usepackage{amsmath} % Equations
    \usepackage{amssymb} % Equations
    \usepackage{textcomp} % defines textquotesingle
    % Hack from http://tex.stackexchange.com/a/47451/13684:
    \AtBeginDocument{%
        \def\PYZsq{\textquotesingle}% Upright quotes in Pygmentized code
    }
    \usepackage{upquote} % Upright quotes for verbatim code
    \usepackage{eurosym} % defines \euro
    \usepackage[mathletters]{ucs} % Extended unicode (utf-8) support
    \usepackage[utf8x]{inputenc} % Allow utf-8 characters in the tex document
    \usepackage{fancyvrb} % verbatim replacement that allows latex
    \usepackage{grffile} % extends the file name processing of package graphics 
                         % to support a larger range 
    % The hyperref package gives us a pdf with properly built
    % internal navigation ('pdf bookmarks' for the table of contents,
    % internal cross-reference links, web links for URLs, etc.)
    \usepackage{hyperref}
    \usepackage{longtable} % longtable support required by pandoc >1.10
    \usepackage{booktabs}  % table support for pandoc > 1.12.2
    \usepackage[inline]{enumitem} % IRkernel/repr support (it uses the enumerate* environment)
    \usepackage[normalem]{ulem} % ulem is needed to support strikethroughs (\sout)
                                % normalem makes italics be italics, not underlines
    

    
    
    % Colors for the hyperref package
    \definecolor{urlcolor}{rgb}{0,.145,.698}
    \definecolor{linkcolor}{rgb}{.71,0.21,0.01}
    \definecolor{citecolor}{rgb}{.12,.54,.11}

    % ANSI colors
    \definecolor{ansi-black}{HTML}{3E424D}
    \definecolor{ansi-black-intense}{HTML}{282C36}
    \definecolor{ansi-red}{HTML}{E75C58}
    \definecolor{ansi-red-intense}{HTML}{B22B31}
    \definecolor{ansi-green}{HTML}{00A250}
    \definecolor{ansi-green-intense}{HTML}{007427}
    \definecolor{ansi-yellow}{HTML}{DDB62B}
    \definecolor{ansi-yellow-intense}{HTML}{B27D12}
    \definecolor{ansi-blue}{HTML}{208FFB}
    \definecolor{ansi-blue-intense}{HTML}{0065CA}
    \definecolor{ansi-magenta}{HTML}{D160C4}
    \definecolor{ansi-magenta-intense}{HTML}{A03196}
    \definecolor{ansi-cyan}{HTML}{60C6C8}
    \definecolor{ansi-cyan-intense}{HTML}{258F8F}
    \definecolor{ansi-white}{HTML}{C5C1B4}
    \definecolor{ansi-white-intense}{HTML}{A1A6B2}

    % commands and environments needed by pandoc snippets
    % extracted from the output of `pandoc -s`
    \providecommand{\tightlist}{%
      \setlength{\itemsep}{0pt}\setlength{\parskip}{0pt}}
    \DefineVerbatimEnvironment{Highlighting}{Verbatim}{commandchars=\\\{\}}
    % Add ',fontsize=\small' for more characters per line
    \newenvironment{Shaded}{}{}
    \newcommand{\KeywordTok}[1]{\textcolor[rgb]{0.00,0.44,0.13}{\textbf{{#1}}}}
    \newcommand{\DataTypeTok}[1]{\textcolor[rgb]{0.56,0.13,0.00}{{#1}}}
    \newcommand{\DecValTok}[1]{\textcolor[rgb]{0.25,0.63,0.44}{{#1}}}
    \newcommand{\BaseNTok}[1]{\textcolor[rgb]{0.25,0.63,0.44}{{#1}}}
    \newcommand{\FloatTok}[1]{\textcolor[rgb]{0.25,0.63,0.44}{{#1}}}
    \newcommand{\CharTok}[1]{\textcolor[rgb]{0.25,0.44,0.63}{{#1}}}
    \newcommand{\StringTok}[1]{\textcolor[rgb]{0.25,0.44,0.63}{{#1}}}
    \newcommand{\CommentTok}[1]{\textcolor[rgb]{0.38,0.63,0.69}{\textit{{#1}}}}
    \newcommand{\OtherTok}[1]{\textcolor[rgb]{0.00,0.44,0.13}{{#1}}}
    \newcommand{\AlertTok}[1]{\textcolor[rgb]{1.00,0.00,0.00}{\textbf{{#1}}}}
    \newcommand{\FunctionTok}[1]{\textcolor[rgb]{0.02,0.16,0.49}{{#1}}}
    \newcommand{\RegionMarkerTok}[1]{{#1}}
    \newcommand{\ErrorTok}[1]{\textcolor[rgb]{1.00,0.00,0.00}{\textbf{{#1}}}}
    \newcommand{\NormalTok}[1]{{#1}}
    
    % Additional commands for more recent versions of Pandoc
    \newcommand{\ConstantTok}[1]{\textcolor[rgb]{0.53,0.00,0.00}{{#1}}}
    \newcommand{\SpecialCharTok}[1]{\textcolor[rgb]{0.25,0.44,0.63}{{#1}}}
    \newcommand{\VerbatimStringTok}[1]{\textcolor[rgb]{0.25,0.44,0.63}{{#1}}}
    \newcommand{\SpecialStringTok}[1]{\textcolor[rgb]{0.73,0.40,0.53}{{#1}}}
    \newcommand{\ImportTok}[1]{{#1}}
    \newcommand{\DocumentationTok}[1]{\textcolor[rgb]{0.73,0.13,0.13}{\textit{{#1}}}}
    \newcommand{\AnnotationTok}[1]{\textcolor[rgb]{0.38,0.63,0.69}{\textbf{\textit{{#1}}}}}
    \newcommand{\CommentVarTok}[1]{\textcolor[rgb]{0.38,0.63,0.69}{\textbf{\textit{{#1}}}}}
    \newcommand{\VariableTok}[1]{\textcolor[rgb]{0.10,0.09,0.49}{{#1}}}
    \newcommand{\ControlFlowTok}[1]{\textcolor[rgb]{0.00,0.44,0.13}{\textbf{{#1}}}}
    \newcommand{\OperatorTok}[1]{\textcolor[rgb]{0.40,0.40,0.40}{{#1}}}
    \newcommand{\BuiltInTok}[1]{{#1}}
    \newcommand{\ExtensionTok}[1]{{#1}}
    \newcommand{\PreprocessorTok}[1]{\textcolor[rgb]{0.74,0.48,0.00}{{#1}}}
    \newcommand{\AttributeTok}[1]{\textcolor[rgb]{0.49,0.56,0.16}{{#1}}}
    \newcommand{\InformationTok}[1]{\textcolor[rgb]{0.38,0.63,0.69}{\textbf{\textit{{#1}}}}}
    \newcommand{\WarningTok}[1]{\textcolor[rgb]{0.38,0.63,0.69}{\textbf{\textit{{#1}}}}}
    
    
    % Define a nice break command that doesn't care if a line doesn't already
    % exist.
    \def\br{\hspace*{\fill} \\* }
    % Math Jax compatability definitions
    \def\gt{>}
    \def\lt{<}
    % Document parameters
    \title{homework2}
    
    
    

    % Pygments definitions
    
\makeatletter
\def\PY@reset{\let\PY@it=\relax \let\PY@bf=\relax%
    \let\PY@ul=\relax \let\PY@tc=\relax%
    \let\PY@bc=\relax \let\PY@ff=\relax}
\def\PY@tok#1{\csname PY@tok@#1\endcsname}
\def\PY@toks#1+{\ifx\relax#1\empty\else%
    \PY@tok{#1}\expandafter\PY@toks\fi}
\def\PY@do#1{\PY@bc{\PY@tc{\PY@ul{%
    \PY@it{\PY@bf{\PY@ff{#1}}}}}}}
\def\PY#1#2{\PY@reset\PY@toks#1+\relax+\PY@do{#2}}

\expandafter\def\csname PY@tok@w\endcsname{\def\PY@tc##1{\textcolor[rgb]{0.73,0.73,0.73}{##1}}}
\expandafter\def\csname PY@tok@c\endcsname{\let\PY@it=\textit\def\PY@tc##1{\textcolor[rgb]{0.25,0.50,0.50}{##1}}}
\expandafter\def\csname PY@tok@cp\endcsname{\def\PY@tc##1{\textcolor[rgb]{0.74,0.48,0.00}{##1}}}
\expandafter\def\csname PY@tok@k\endcsname{\let\PY@bf=\textbf\def\PY@tc##1{\textcolor[rgb]{0.00,0.50,0.00}{##1}}}
\expandafter\def\csname PY@tok@kp\endcsname{\def\PY@tc##1{\textcolor[rgb]{0.00,0.50,0.00}{##1}}}
\expandafter\def\csname PY@tok@kt\endcsname{\def\PY@tc##1{\textcolor[rgb]{0.69,0.00,0.25}{##1}}}
\expandafter\def\csname PY@tok@o\endcsname{\def\PY@tc##1{\textcolor[rgb]{0.40,0.40,0.40}{##1}}}
\expandafter\def\csname PY@tok@ow\endcsname{\let\PY@bf=\textbf\def\PY@tc##1{\textcolor[rgb]{0.67,0.13,1.00}{##1}}}
\expandafter\def\csname PY@tok@nb\endcsname{\def\PY@tc##1{\textcolor[rgb]{0.00,0.50,0.00}{##1}}}
\expandafter\def\csname PY@tok@nf\endcsname{\def\PY@tc##1{\textcolor[rgb]{0.00,0.00,1.00}{##1}}}
\expandafter\def\csname PY@tok@nc\endcsname{\let\PY@bf=\textbf\def\PY@tc##1{\textcolor[rgb]{0.00,0.00,1.00}{##1}}}
\expandafter\def\csname PY@tok@nn\endcsname{\let\PY@bf=\textbf\def\PY@tc##1{\textcolor[rgb]{0.00,0.00,1.00}{##1}}}
\expandafter\def\csname PY@tok@ne\endcsname{\let\PY@bf=\textbf\def\PY@tc##1{\textcolor[rgb]{0.82,0.25,0.23}{##1}}}
\expandafter\def\csname PY@tok@nv\endcsname{\def\PY@tc##1{\textcolor[rgb]{0.10,0.09,0.49}{##1}}}
\expandafter\def\csname PY@tok@no\endcsname{\def\PY@tc##1{\textcolor[rgb]{0.53,0.00,0.00}{##1}}}
\expandafter\def\csname PY@tok@nl\endcsname{\def\PY@tc##1{\textcolor[rgb]{0.63,0.63,0.00}{##1}}}
\expandafter\def\csname PY@tok@ni\endcsname{\let\PY@bf=\textbf\def\PY@tc##1{\textcolor[rgb]{0.60,0.60,0.60}{##1}}}
\expandafter\def\csname PY@tok@na\endcsname{\def\PY@tc##1{\textcolor[rgb]{0.49,0.56,0.16}{##1}}}
\expandafter\def\csname PY@tok@nt\endcsname{\let\PY@bf=\textbf\def\PY@tc##1{\textcolor[rgb]{0.00,0.50,0.00}{##1}}}
\expandafter\def\csname PY@tok@nd\endcsname{\def\PY@tc##1{\textcolor[rgb]{0.67,0.13,1.00}{##1}}}
\expandafter\def\csname PY@tok@s\endcsname{\def\PY@tc##1{\textcolor[rgb]{0.73,0.13,0.13}{##1}}}
\expandafter\def\csname PY@tok@sd\endcsname{\let\PY@it=\textit\def\PY@tc##1{\textcolor[rgb]{0.73,0.13,0.13}{##1}}}
\expandafter\def\csname PY@tok@si\endcsname{\let\PY@bf=\textbf\def\PY@tc##1{\textcolor[rgb]{0.73,0.40,0.53}{##1}}}
\expandafter\def\csname PY@tok@se\endcsname{\let\PY@bf=\textbf\def\PY@tc##1{\textcolor[rgb]{0.73,0.40,0.13}{##1}}}
\expandafter\def\csname PY@tok@sr\endcsname{\def\PY@tc##1{\textcolor[rgb]{0.73,0.40,0.53}{##1}}}
\expandafter\def\csname PY@tok@ss\endcsname{\def\PY@tc##1{\textcolor[rgb]{0.10,0.09,0.49}{##1}}}
\expandafter\def\csname PY@tok@sx\endcsname{\def\PY@tc##1{\textcolor[rgb]{0.00,0.50,0.00}{##1}}}
\expandafter\def\csname PY@tok@m\endcsname{\def\PY@tc##1{\textcolor[rgb]{0.40,0.40,0.40}{##1}}}
\expandafter\def\csname PY@tok@gh\endcsname{\let\PY@bf=\textbf\def\PY@tc##1{\textcolor[rgb]{0.00,0.00,0.50}{##1}}}
\expandafter\def\csname PY@tok@gu\endcsname{\let\PY@bf=\textbf\def\PY@tc##1{\textcolor[rgb]{0.50,0.00,0.50}{##1}}}
\expandafter\def\csname PY@tok@gd\endcsname{\def\PY@tc##1{\textcolor[rgb]{0.63,0.00,0.00}{##1}}}
\expandafter\def\csname PY@tok@gi\endcsname{\def\PY@tc##1{\textcolor[rgb]{0.00,0.63,0.00}{##1}}}
\expandafter\def\csname PY@tok@gr\endcsname{\def\PY@tc##1{\textcolor[rgb]{1.00,0.00,0.00}{##1}}}
\expandafter\def\csname PY@tok@ge\endcsname{\let\PY@it=\textit}
\expandafter\def\csname PY@tok@gs\endcsname{\let\PY@bf=\textbf}
\expandafter\def\csname PY@tok@gp\endcsname{\let\PY@bf=\textbf\def\PY@tc##1{\textcolor[rgb]{0.00,0.00,0.50}{##1}}}
\expandafter\def\csname PY@tok@go\endcsname{\def\PY@tc##1{\textcolor[rgb]{0.53,0.53,0.53}{##1}}}
\expandafter\def\csname PY@tok@gt\endcsname{\def\PY@tc##1{\textcolor[rgb]{0.00,0.27,0.87}{##1}}}
\expandafter\def\csname PY@tok@err\endcsname{\def\PY@bc##1{\setlength{\fboxsep}{0pt}\fcolorbox[rgb]{1.00,0.00,0.00}{1,1,1}{\strut ##1}}}
\expandafter\def\csname PY@tok@kc\endcsname{\let\PY@bf=\textbf\def\PY@tc##1{\textcolor[rgb]{0.00,0.50,0.00}{##1}}}
\expandafter\def\csname PY@tok@kd\endcsname{\let\PY@bf=\textbf\def\PY@tc##1{\textcolor[rgb]{0.00,0.50,0.00}{##1}}}
\expandafter\def\csname PY@tok@kn\endcsname{\let\PY@bf=\textbf\def\PY@tc##1{\textcolor[rgb]{0.00,0.50,0.00}{##1}}}
\expandafter\def\csname PY@tok@kr\endcsname{\let\PY@bf=\textbf\def\PY@tc##1{\textcolor[rgb]{0.00,0.50,0.00}{##1}}}
\expandafter\def\csname PY@tok@bp\endcsname{\def\PY@tc##1{\textcolor[rgb]{0.00,0.50,0.00}{##1}}}
\expandafter\def\csname PY@tok@fm\endcsname{\def\PY@tc##1{\textcolor[rgb]{0.00,0.00,1.00}{##1}}}
\expandafter\def\csname PY@tok@vc\endcsname{\def\PY@tc##1{\textcolor[rgb]{0.10,0.09,0.49}{##1}}}
\expandafter\def\csname PY@tok@vg\endcsname{\def\PY@tc##1{\textcolor[rgb]{0.10,0.09,0.49}{##1}}}
\expandafter\def\csname PY@tok@vi\endcsname{\def\PY@tc##1{\textcolor[rgb]{0.10,0.09,0.49}{##1}}}
\expandafter\def\csname PY@tok@vm\endcsname{\def\PY@tc##1{\textcolor[rgb]{0.10,0.09,0.49}{##1}}}
\expandafter\def\csname PY@tok@sa\endcsname{\def\PY@tc##1{\textcolor[rgb]{0.73,0.13,0.13}{##1}}}
\expandafter\def\csname PY@tok@sb\endcsname{\def\PY@tc##1{\textcolor[rgb]{0.73,0.13,0.13}{##1}}}
\expandafter\def\csname PY@tok@sc\endcsname{\def\PY@tc##1{\textcolor[rgb]{0.73,0.13,0.13}{##1}}}
\expandafter\def\csname PY@tok@dl\endcsname{\def\PY@tc##1{\textcolor[rgb]{0.73,0.13,0.13}{##1}}}
\expandafter\def\csname PY@tok@s2\endcsname{\def\PY@tc##1{\textcolor[rgb]{0.73,0.13,0.13}{##1}}}
\expandafter\def\csname PY@tok@sh\endcsname{\def\PY@tc##1{\textcolor[rgb]{0.73,0.13,0.13}{##1}}}
\expandafter\def\csname PY@tok@s1\endcsname{\def\PY@tc##1{\textcolor[rgb]{0.73,0.13,0.13}{##1}}}
\expandafter\def\csname PY@tok@mb\endcsname{\def\PY@tc##1{\textcolor[rgb]{0.40,0.40,0.40}{##1}}}
\expandafter\def\csname PY@tok@mf\endcsname{\def\PY@tc##1{\textcolor[rgb]{0.40,0.40,0.40}{##1}}}
\expandafter\def\csname PY@tok@mh\endcsname{\def\PY@tc##1{\textcolor[rgb]{0.40,0.40,0.40}{##1}}}
\expandafter\def\csname PY@tok@mi\endcsname{\def\PY@tc##1{\textcolor[rgb]{0.40,0.40,0.40}{##1}}}
\expandafter\def\csname PY@tok@il\endcsname{\def\PY@tc##1{\textcolor[rgb]{0.40,0.40,0.40}{##1}}}
\expandafter\def\csname PY@tok@mo\endcsname{\def\PY@tc##1{\textcolor[rgb]{0.40,0.40,0.40}{##1}}}
\expandafter\def\csname PY@tok@ch\endcsname{\let\PY@it=\textit\def\PY@tc##1{\textcolor[rgb]{0.25,0.50,0.50}{##1}}}
\expandafter\def\csname PY@tok@cm\endcsname{\let\PY@it=\textit\def\PY@tc##1{\textcolor[rgb]{0.25,0.50,0.50}{##1}}}
\expandafter\def\csname PY@tok@cpf\endcsname{\let\PY@it=\textit\def\PY@tc##1{\textcolor[rgb]{0.25,0.50,0.50}{##1}}}
\expandafter\def\csname PY@tok@c1\endcsname{\let\PY@it=\textit\def\PY@tc##1{\textcolor[rgb]{0.25,0.50,0.50}{##1}}}
\expandafter\def\csname PY@tok@cs\endcsname{\let\PY@it=\textit\def\PY@tc##1{\textcolor[rgb]{0.25,0.50,0.50}{##1}}}

\def\PYZbs{\char`\\}
\def\PYZus{\char`\_}
\def\PYZob{\char`\{}
\def\PYZcb{\char`\}}
\def\PYZca{\char`\^}
\def\PYZam{\char`\&}
\def\PYZlt{\char`\<}
\def\PYZgt{\char`\>}
\def\PYZsh{\char`\#}
\def\PYZpc{\char`\%}
\def\PYZdl{\char`\$}
\def\PYZhy{\char`\-}
\def\PYZsq{\char`\'}
\def\PYZdq{\char`\"}
\def\PYZti{\char`\~}
% for compatibility with earlier versions
\def\PYZat{@}
\def\PYZlb{[}
\def\PYZrb{]}
\makeatother


    % Exact colors from NB
    \definecolor{incolor}{rgb}{0.0, 0.0, 0.5}
    \definecolor{outcolor}{rgb}{0.545, 0.0, 0.0}



    
    % Prevent overflowing lines due to hard-to-break entities
    \sloppy 
    % Setup hyperref package
    \hypersetup{
      breaklinks=true,  % so long urls are correctly broken across lines
      colorlinks=true,
      urlcolor=urlcolor,
      linkcolor=linkcolor,
      citecolor=citecolor,
      }
    % Slightly bigger margins than the latex defaults
    
    \geometry{verbose,tmargin=1in,bmargin=1in,lmargin=1in,rmargin=1in}
    
    

    \begin{document}
    
    
    \maketitle
    
    

    
    \hypertarget{homework-2---berkeley-stat-157}{%
\section{Homework 2 - Berkeley STAT
157}\label{homework-2---berkeley-stat-157}}

Handout 1/29/2019, due 2/5/2019 by 4pm in Git by committing to your
repository.

    \begin{Verbatim}[commandchars=\\\{\}]
{\color{incolor}In [{\color{incolor}123}]:} \PY{k+kn}{from} \PY{n+nn}{mxnet} \PY{k}{import} \PY{n}{nd}\PY{p}{,} \PY{n}{autograd}\PY{p}{,} \PY{n}{gluon}
          \PY{k+kn}{from} \PY{n+nn}{mxnet} \PY{k}{import} \PY{n}{ndarray} \PY{k}{as} \PY{n}{nd}
          \PY{k+kn}{import} \PY{n+nn}{numpy} \PY{k}{as} \PY{n+nn}{np}
          \PY{k+kn}{import} \PY{n+nn}{mxnet} \PY{k}{as} \PY{n+nn}{mx}
          \PY{k+kn}{from} \PY{n+nn}{mxnet} \PY{k}{import} \PY{n}{nd}
          \PY{k+kn}{from} \PY{n+nn}{mxnet}\PY{n+nn}{.}\PY{n+nn}{gluon} \PY{k}{import} \PY{n}{nn}
          \PY{k+kn}{from} \PY{n+nn}{functools} \PY{k}{import} \PY{n}{reduce}
          \PY{k+kn}{from} \PY{n+nn}{operator} \PY{k}{import} \PY{n}{mul}
          \PY{k+kn}{from} \PY{n+nn}{matplotlib} \PY{k}{import} \PY{n}{pyplot} \PY{k}{as} \PY{n}{plt}
\end{Verbatim}


    \hypertarget{multinomial-sampling}{%
\section{1. Multinomial Sampling}\label{multinomial-sampling}}

Implement a sampler from a discrete distribution from scratch, mimicking
the function \texttt{mxnet.ndarray.random.multinomial}. Its arguments
should be a vector of probabilities \(p\). You can assume that the
probabilities are normalized, i.e.~tha they sum up to \(1\). Make the
call signature as follows:

\begin{verbatim}
samples = sampler(probs, shape) 

probs   : An ndarray vector of size n of nonnegative numbers summing up to 1
shape   : A list of dimensions for the output
samples : Samples from probs with shape matching shape
\end{verbatim}

Hints:

\begin{enumerate}
\def\labelenumi{\arabic{enumi}.}
\tightlist
\item
  Use \texttt{mxnet.ndarray.random.uniform} to get a sample from
  \(U[0,1]\).
\item
  You can simplify things for \texttt{probs} by computing the cumulative
  sum over \texttt{probs}.
\end{enumerate}

    \begin{Verbatim}[commandchars=\\\{\}]
{\color{incolor}In [{\color{incolor}86}]:} \PY{k}{def} \PY{n+nf}{sampler}\PY{p}{(}\PY{n}{probs}\PY{p}{,} \PY{n}{shape}\PY{p}{)}\PY{p}{:}
             \PY{c+c1}{\PYZsh{}\PYZsh{} Add your codes here}
             \PY{n}{cum\PYZus{}probs} \PY{o}{=} \PY{n}{np}\PY{o}{.}\PY{n}{cumsum}\PY{p}{(}\PY{n}{probs}\PY{o}{.}\PY{n}{asnumpy}\PY{p}{(}\PY{p}{)}\PY{p}{)}
             
             \PY{n}{total} \PY{o}{=} \PY{n}{reduce}\PY{p}{(}\PY{n}{mul}\PY{p}{,} \PY{n}{shape}\PY{p}{,} \PY{l+m+mi}{1}\PY{p}{)}
             \PY{n}{result} \PY{o}{=} \PY{n}{nd}\PY{o}{.}\PY{n}{zeros}\PY{p}{(}\PY{p}{(}\PY{l+m+mi}{1}\PY{p}{,} \PY{n}{total}\PY{p}{)}\PY{p}{)}
             
             \PY{k}{for} \PY{n}{i} \PY{o+ow}{in} \PY{n+nb}{range}\PY{p}{(}\PY{n}{total}\PY{p}{)}\PY{p}{:}
                 \PY{n}{temp} \PY{o}{=} \PY{n}{nd}\PY{o}{.}\PY{n}{random}\PY{o}{.}\PY{n}{uniform}\PY{p}{(}\PY{p}{)}
                 \PY{k}{for} \PY{n}{j} \PY{o+ow}{in} \PY{n+nb}{range}\PY{p}{(}\PY{n+nb}{len}\PY{p}{(}\PY{n}{cum\PYZus{}probs}\PY{p}{)}\PY{p}{)}\PY{p}{:}
                     \PY{k}{if} \PY{n}{cum\PYZus{}probs}\PY{p}{[}\PY{n}{j}\PY{p}{]} \PY{o}{\PYZgt{}} \PY{n}{temp}\PY{p}{:}
                         \PY{n}{result}\PY{p}{[}\PY{l+m+mi}{0}\PY{p}{,} \PY{n}{i}\PY{p}{]} \PY{o}{=} \PY{n}{j}
                         \PY{k}{break}
             
             \PY{n}{result} \PY{o}{=} \PY{n}{result}\PY{o}{.}\PY{n}{reshape}\PY{p}{(}\PY{n}{shape}\PY{p}{)}
             \PY{k}{return} \PY{n}{result}
         
         \PY{c+c1}{\PYZsh{} a simple test}
         
         \PY{n}{sampler}\PY{p}{(}\PY{n}{nd}\PY{o}{.}\PY{n}{array}\PY{p}{(}\PY{p}{[}\PY{l+m+mf}{0.2}\PY{p}{,} \PY{l+m+mf}{0.3}\PY{p}{,} \PY{l+m+mf}{0.5}\PY{p}{]}\PY{p}{)}\PY{p}{,} \PY{p}{(}\PY{l+m+mi}{2}\PY{p}{,}\PY{l+m+mi}{3}\PY{p}{)}\PY{p}{)}
\end{Verbatim}


\begin{Verbatim}[commandchars=\\\{\}]
{\color{outcolor}Out[{\color{outcolor}86}]:} 
         [[2. 2. 1.]
          [2. 2. 1.]]
         <NDArray 2x3 @cpu(0)>
\end{Verbatim}
            
    \hypertarget{central-limit-theorem}{%
\section{2. Central Limit Theorem}\label{central-limit-theorem}}

Let's explore the Central Limit Theorem when applied to text processing.

\begin{itemize}
\tightlist
\item
  Download
  \href{https://www.gutenberg.org/files/84/84-0.txt}{https://www.gutenberg.org/ebooks/84}
  from Project Gutenberg
\item
  Remove punctuation, uppercase / lowercase, and split the text up into
  individual tokens (words).
\item
  For the words \texttt{a}, \texttt{and}, \texttt{the}, \texttt{i},
  \texttt{is} compute their respective counts as the book progresses,
  i.e. \[n_\mathrm{the}[i] = \sum_{j = 1}^i \{w_j = \mathrm{the}\}\]
\item
  Plot the proportions \(n_\mathrm{word}[i] / i\) over the document in
  one plot.
\item
  Find an envelope of the shape \(O(1/\sqrt{i})\) for each of these five
  words. (Hint, check the last page of the
  \href{http://courses.d2l.ai/berkeley-stat-157/slides/1_24/sampling.pdf}{sampling
  notebook})
\item
  Why can we \textbf{not} apply the Central Limit Theorem directly?
\item
  How would we have to change the text for it to apply?
\item
  Why does it still work quite well?
\end{itemize}

    \begin{Verbatim}[commandchars=\\\{\}]
{\color{incolor}In [{\color{incolor}131}]:} \PY{k+kn}{import} \PY{n+nn}{re}
          
          \PY{n}{filename} \PY{o}{=} \PY{n}{gluon}\PY{o}{.}\PY{n}{utils}\PY{o}{.}\PY{n}{download}\PY{p}{(}\PY{l+s+s1}{\PYZsq{}}\PY{l+s+s1}{https://www.gutenberg.org/files/84/84\PYZhy{}0.txt}\PY{l+s+s1}{\PYZsq{}}\PY{p}{)}
          \PY{k}{with} \PY{n+nb}{open}\PY{p}{(}\PY{n}{filename}\PY{p}{)} \PY{k}{as} \PY{n}{f}\PY{p}{:}
              \PY{n}{book} \PY{o}{=} \PY{n}{f}\PY{o}{.}\PY{n}{read}\PY{p}{(}\PY{p}{)}
          
          \PY{c+c1}{\PYZsh{}\PYZsh{} Add your codes here}
          \PY{n}{book} \PY{o}{=} \PY{n}{book}\PY{o}{.}\PY{n}{lower}\PY{p}{(}\PY{p}{)}
          \PY{n}{book} \PY{o}{=} \PY{n}{re}\PY{o}{.}\PY{n}{sub}\PY{p}{(}\PY{l+s+sa}{r}\PY{l+s+s1}{\PYZsq{}}\PY{l+s+s1}{[\PYZca{}}\PY{l+s+s1}{\PYZbs{}}\PY{l+s+s1}{w}\PY{l+s+s1}{\PYZbs{}}\PY{l+s+s1}{s]}\PY{l+s+s1}{\PYZsq{}}\PY{p}{,}\PY{l+s+s1}{\PYZsq{}}\PY{l+s+s1}{\PYZsq{}}\PY{p}{,}\PY{n}{book}\PY{p}{)}
          
          \PY{n}{words} \PY{o}{=} \PY{n}{book}\PY{o}{.}\PY{n}{split}\PY{p}{(}\PY{p}{)}
          
          \PY{n}{count} \PY{o}{=} \PY{l+m+mi}{0}
          \PY{n}{a\PYZus{}counts} \PY{o}{=} \PY{n}{np}\PY{o}{.}\PY{n}{zeros}\PY{p}{(}\PY{p}{(}\PY{l+m+mi}{1}\PY{p}{,} \PY{n+nb}{len}\PY{p}{(}\PY{n}{words}\PY{p}{)}\PY{p}{)}\PY{p}{)}
          \PY{n}{and\PYZus{}counts} \PY{o}{=} \PY{n}{np}\PY{o}{.}\PY{n}{zeros}\PY{p}{(}\PY{p}{(}\PY{l+m+mi}{1}\PY{p}{,} \PY{n+nb}{len}\PY{p}{(}\PY{n}{words}\PY{p}{)}\PY{p}{)}\PY{p}{)}
          \PY{n}{the\PYZus{}counts} \PY{o}{=} \PY{n}{np}\PY{o}{.}\PY{n}{zeros}\PY{p}{(}\PY{p}{(}\PY{l+m+mi}{1}\PY{p}{,} \PY{n+nb}{len}\PY{p}{(}\PY{n}{words}\PY{p}{)}\PY{p}{)}\PY{p}{)}
          \PY{n}{i\PYZus{}counts} \PY{o}{=} \PY{n}{np}\PY{o}{.}\PY{n}{zeros}\PY{p}{(}\PY{p}{(}\PY{l+m+mi}{1}\PY{p}{,} \PY{n+nb}{len}\PY{p}{(}\PY{n}{words}\PY{p}{)}\PY{p}{)}\PY{p}{)}
          \PY{n}{is\PYZus{}counts} \PY{o}{=} \PY{n}{np}\PY{o}{.}\PY{n}{zeros}\PY{p}{(}\PY{p}{(}\PY{l+m+mi}{1}\PY{p}{,} \PY{n+nb}{len}\PY{p}{(}\PY{n}{words}\PY{p}{)}\PY{p}{)}\PY{p}{)}
          
          \PY{k}{for} \PY{n}{i} \PY{o+ow}{in} \PY{n+nb}{range}\PY{p}{(}\PY{l+m+mi}{1}\PY{p}{,} \PY{n+nb}{len}\PY{p}{(}\PY{n}{words}\PY{p}{)}\PY{p}{)}\PY{p}{:}
              \PY{n}{count} \PY{o}{+}\PY{o}{=} \PY{l+m+mi}{1}
              \PY{k}{if} \PY{n}{words}\PY{p}{[}\PY{n}{i}\PY{p}{]} \PY{o}{==} \PY{l+s+s2}{\PYZdq{}}\PY{l+s+s2}{a}\PY{l+s+s2}{\PYZdq{}}\PY{p}{:}
                  \PY{n}{a\PYZus{}counts}\PY{p}{[}\PY{l+m+mi}{0}\PY{p}{,} \PY{n}{i}\PY{p}{]} \PY{o}{=} \PY{n}{a\PYZus{}counts}\PY{p}{[}\PY{l+m+mi}{0}\PY{p}{,}\PY{n}{i} \PY{o}{\PYZhy{}} \PY{l+m+mi}{1}\PY{p}{]} \PY{o}{+} \PY{l+m+mi}{1}
              \PY{k}{else}\PY{p}{:}
                  \PY{n}{a\PYZus{}counts}\PY{p}{[}\PY{l+m+mi}{0}\PY{p}{,} \PY{n}{i}\PY{p}{]} \PY{o}{=} \PY{n}{a\PYZus{}counts}\PY{p}{[}\PY{l+m+mi}{0}\PY{p}{,}\PY{n}{i} \PY{o}{\PYZhy{}} \PY{l+m+mi}{1}\PY{p}{]}
                  
              \PY{k}{if} \PY{n}{words}\PY{p}{[}\PY{n}{i}\PY{p}{]} \PY{o}{==} \PY{l+s+s2}{\PYZdq{}}\PY{l+s+s2}{and}\PY{l+s+s2}{\PYZdq{}}\PY{p}{:}
                  \PY{n}{and\PYZus{}counts}\PY{p}{[}\PY{l+m+mi}{0}\PY{p}{,}\PY{n}{i}\PY{p}{]} \PY{o}{=} \PY{n}{and\PYZus{}counts}\PY{p}{[}\PY{l+m+mi}{0}\PY{p}{,}\PY{n}{i} \PY{o}{\PYZhy{}} \PY{l+m+mi}{1}\PY{p}{]} \PY{o}{+} \PY{l+m+mi}{1}
              \PY{k}{else}\PY{p}{:}
                  \PY{n}{and\PYZus{}counts}\PY{p}{[}\PY{l+m+mi}{0}\PY{p}{,}\PY{n}{i}\PY{p}{]} \PY{o}{=} \PY{n}{and\PYZus{}counts}\PY{p}{[}\PY{l+m+mi}{0}\PY{p}{,}\PY{n}{i} \PY{o}{\PYZhy{}} \PY{l+m+mi}{1}\PY{p}{]}
                  
              \PY{k}{if} \PY{n}{words}\PY{p}{[}\PY{n}{i}\PY{p}{]} \PY{o}{==} \PY{l+s+s2}{\PYZdq{}}\PY{l+s+s2}{the}\PY{l+s+s2}{\PYZdq{}}\PY{p}{:}
                  \PY{n}{the\PYZus{}counts}\PY{p}{[}\PY{l+m+mi}{0}\PY{p}{,}\PY{n}{i}\PY{p}{]} \PY{o}{=} \PY{n}{the\PYZus{}counts}\PY{p}{[}\PY{l+m+mi}{0}\PY{p}{,}\PY{n}{i} \PY{o}{\PYZhy{}} \PY{l+m+mi}{1}\PY{p}{]} \PY{o}{+} \PY{l+m+mi}{1}
              \PY{k}{else}\PY{p}{:}
                  \PY{n}{the\PYZus{}counts}\PY{p}{[}\PY{l+m+mi}{0}\PY{p}{,}\PY{n}{i}\PY{p}{]} \PY{o}{=} \PY{n}{the\PYZus{}counts}\PY{p}{[}\PY{l+m+mi}{0}\PY{p}{,}\PY{n}{i} \PY{o}{\PYZhy{}} \PY{l+m+mi}{1}\PY{p}{]}
                  
              \PY{k}{if} \PY{n}{words}\PY{p}{[}\PY{n}{i}\PY{p}{]} \PY{o}{==} \PY{l+s+s2}{\PYZdq{}}\PY{l+s+s2}{i}\PY{l+s+s2}{\PYZdq{}}\PY{p}{:}
                  \PY{n}{i\PYZus{}counts}\PY{p}{[}\PY{l+m+mi}{0}\PY{p}{,}\PY{n}{i}\PY{p}{]} \PY{o}{=} \PY{n}{i\PYZus{}counts}\PY{p}{[}\PY{l+m+mi}{0}\PY{p}{,}\PY{n}{i} \PY{o}{\PYZhy{}} \PY{l+m+mi}{1}\PY{p}{]} \PY{o}{+} \PY{l+m+mi}{1}
              \PY{k}{else}\PY{p}{:}
                  \PY{n}{i\PYZus{}counts}\PY{p}{[}\PY{l+m+mi}{0}\PY{p}{,}\PY{n}{i}\PY{p}{]} \PY{o}{=} \PY{n}{i\PYZus{}counts}\PY{p}{[}\PY{l+m+mi}{0}\PY{p}{,}\PY{n}{i} \PY{o}{\PYZhy{}} \PY{l+m+mi}{1}\PY{p}{]}
                  
              \PY{k}{if} \PY{n}{words}\PY{p}{[}\PY{n}{i}\PY{p}{]} \PY{o}{==} \PY{l+s+s2}{\PYZdq{}}\PY{l+s+s2}{is}\PY{l+s+s2}{\PYZdq{}}\PY{p}{:}
                  \PY{n}{is\PYZus{}counts}\PY{p}{[}\PY{l+m+mi}{0}\PY{p}{,}\PY{n}{i}\PY{p}{]} \PY{o}{=} \PY{n}{is\PYZus{}counts}\PY{p}{[}\PY{l+m+mi}{0}\PY{p}{,}\PY{n}{i} \PY{o}{\PYZhy{}} \PY{l+m+mi}{1}\PY{p}{]} \PY{o}{+} \PY{l+m+mi}{1}
              \PY{k}{else}\PY{p}{:}
                  \PY{n}{is\PYZus{}counts}\PY{p}{[}\PY{l+m+mi}{0}\PY{p}{,}\PY{n}{i}\PY{p}{]} \PY{o}{=} \PY{n}{is\PYZus{}counts}\PY{p}{[}\PY{l+m+mi}{0}\PY{p}{,}\PY{n}{i} \PY{o}{\PYZhy{}} \PY{l+m+mi}{1}\PY{p}{]}
\end{Verbatim}


    Plot the proportions of the words:

    \begin{Verbatim}[commandchars=\\\{\}]
{\color{incolor}In [{\color{incolor}130}]:} \PY{n}{y} \PY{o}{=} \PY{n}{np}\PY{o}{.}\PY{n}{arange}\PY{p}{(}\PY{l+m+mi}{1}\PY{p}{,} \PY{n+nb}{len}\PY{p}{(}\PY{n}{words}\PY{p}{)}\PY{o}{+}\PY{l+m+mi}{1}\PY{p}{)}\PY{o}{.}\PY{n}{reshape}\PY{p}{(}\PY{n+nb}{len}\PY{p}{(}\PY{n}{words}\PY{p}{)}\PY{p}{,}\PY{l+m+mi}{1}\PY{p}{)}
          \PY{n}{a\PYZus{}props} \PY{o}{=} \PY{n}{np}\PY{o}{.}\PY{n}{zeros}\PY{p}{(}\PY{p}{(}\PY{l+m+mi}{1}\PY{p}{,} \PY{n+nb}{len}\PY{p}{(}\PY{n}{words}\PY{p}{)}\PY{p}{)}\PY{p}{)}
          \PY{n}{and\PYZus{}props} \PY{o}{=} \PY{n}{np}\PY{o}{.}\PY{n}{zeros}\PY{p}{(}\PY{p}{(}\PY{l+m+mi}{1}\PY{p}{,} \PY{n+nb}{len}\PY{p}{(}\PY{n}{words}\PY{p}{)}\PY{p}{)}\PY{p}{)}
          \PY{n}{the\PYZus{}props} \PY{o}{=} \PY{n}{np}\PY{o}{.}\PY{n}{zeros}\PY{p}{(}\PY{p}{(}\PY{l+m+mi}{1}\PY{p}{,} \PY{n+nb}{len}\PY{p}{(}\PY{n}{words}\PY{p}{)}\PY{p}{)}\PY{p}{)}
          \PY{n}{i\PYZus{}props} \PY{o}{=} \PY{n}{np}\PY{o}{.}\PY{n}{zeros}\PY{p}{(}\PY{p}{(}\PY{l+m+mi}{1}\PY{p}{,} \PY{n+nb}{len}\PY{p}{(}\PY{n}{words}\PY{p}{)}\PY{p}{)}\PY{p}{)}
          \PY{n}{is\PYZus{}props} \PY{o}{=} \PY{n}{np}\PY{o}{.}\PY{n}{zeros}\PY{p}{(}\PY{p}{(}\PY{l+m+mi}{1}\PY{p}{,} \PY{n+nb}{len}\PY{p}{(}\PY{n}{words}\PY{p}{)}\PY{p}{)}\PY{p}{)}
          
          \PY{k}{for} \PY{n}{i} \PY{o+ow}{in} \PY{n+nb}{range}\PY{p}{(}\PY{l+m+mi}{1}\PY{p}{,} \PY{n}{count}\PY{o}{+}\PY{l+m+mi}{1}\PY{p}{)}\PY{p}{:}
              \PY{n}{a\PYZus{}props}\PY{p}{[}\PY{l+m+mi}{0}\PY{p}{,} \PY{n}{i}\PY{p}{]} \PY{o}{=} \PY{n}{a\PYZus{}counts}\PY{p}{[}\PY{l+m+mi}{0}\PY{p}{,} \PY{n}{i}\PY{p}{]}\PY{o}{/}\PY{n}{i}
              \PY{n}{and\PYZus{}props}\PY{p}{[}\PY{l+m+mi}{0}\PY{p}{,} \PY{n}{i}\PY{p}{]} \PY{o}{=} \PY{n}{and\PYZus{}counts}\PY{p}{[}\PY{l+m+mi}{0}\PY{p}{,} \PY{n}{i}\PY{p}{]}\PY{o}{/}\PY{n}{i}
              \PY{n}{the\PYZus{}props}\PY{p}{[}\PY{l+m+mi}{0}\PY{p}{,} \PY{n}{i}\PY{p}{]} \PY{o}{=} \PY{n}{the\PYZus{}counts}\PY{p}{[}\PY{l+m+mi}{0}\PY{p}{,} \PY{n}{i}\PY{p}{]}\PY{o}{/}\PY{n}{i}
              \PY{n}{i\PYZus{}props}\PY{p}{[}\PY{l+m+mi}{0}\PY{p}{,} \PY{n}{i}\PY{p}{]} \PY{o}{=} \PY{n}{i\PYZus{}counts}\PY{p}{[}\PY{l+m+mi}{0}\PY{p}{,} \PY{n}{i}\PY{p}{]}\PY{o}{/}\PY{n}{i}
              \PY{n}{is\PYZus{}props}\PY{p}{[}\PY{l+m+mi}{0}\PY{p}{,} \PY{n}{i}\PY{p}{]} \PY{o}{=} \PY{n}{is\PYZus{}counts}\PY{p}{[}\PY{l+m+mi}{0}\PY{p}{,} \PY{n}{i}\PY{p}{]}\PY{o}{/}\PY{n}{i}
          
          \PY{n}{plt}\PY{o}{.}\PY{n}{figure}\PY{p}{(}\PY{n}{figsize}\PY{o}{=}\PY{p}{(}\PY{l+m+mi}{10}\PY{p}{,}\PY{l+m+mi}{5}\PY{p}{)}\PY{p}{)}
          \PY{n}{plt}\PY{o}{.}\PY{n}{semilogx}\PY{p}{(}\PY{n}{y}\PY{p}{,}\PY{n}{a\PYZus{}props}\PY{p}{[}\PY{l+m+mi}{0}\PY{p}{,}\PY{p}{:}\PY{p}{]}\PY{p}{)}
          \PY{n}{plt}\PY{o}{.}\PY{n}{semilogx}\PY{p}{(}\PY{n}{y}\PY{p}{,}\PY{n}{and\PYZus{}props}\PY{p}{[}\PY{l+m+mi}{0}\PY{p}{,}\PY{p}{:}\PY{p}{]}\PY{p}{)} 
          \PY{n}{plt}\PY{o}{.}\PY{n}{semilogx}\PY{p}{(}\PY{n}{y}\PY{p}{,}\PY{n}{the\PYZus{}props}\PY{p}{[}\PY{l+m+mi}{0}\PY{p}{,}\PY{p}{:}\PY{p}{]}\PY{p}{)} 
          \PY{n}{plt}\PY{o}{.}\PY{n}{semilogx}\PY{p}{(}\PY{n}{y}\PY{p}{,}\PY{n}{i\PYZus{}props}\PY{p}{[}\PY{l+m+mi}{0}\PY{p}{,}\PY{p}{:}\PY{p}{]}\PY{p}{)} 
          \PY{n}{plt}\PY{o}{.}\PY{n}{semilogx}\PY{p}{(}\PY{n}{y}\PY{p}{,}\PY{n}{is\PYZus{}props}\PY{p}{[}\PY{l+m+mi}{0}\PY{p}{,}\PY{p}{:}\PY{p}{]}\PY{p}{)} 
          \PY{n}{plt}\PY{o}{.}\PY{n}{show}\PY{p}{(}\PY{p}{)}
\end{Verbatim}


    \begin{center}
    \adjustimage{max size={0.9\linewidth}{0.9\paperheight}}{output_7_0.png}
    \end{center}
    { \hspace*{\fill} \\}
    
    We cannot apply CLT directly because we don't know if the words are
independent in this text file, so the proportion of the words may not
perform as the central limit theorem predicts. To solve this, we can
totally randomize the text file, break the order of the words, so that
the central limit theorem only focus on the frequency of the word, but
not the dependence to some other words. It still works because Central
limit theorem ensures that as long as the dataset is large enough, the
frequency of the data will converge to a certain point, and randomized
data also works in this condition.

    \hypertarget{denominator-layout-notation}{%
\subsection{3. Denominator-layout
notation}\label{denominator-layout-notation}}

We used the numerator-layout notation for matrix calculus in class, now
let's examine the denominator-layout notation.

Given \(x, y\in\mathbb R\), \(\mathbf x\in\mathbb R^n\) and
\(\mathbf y \in \mathbb R^m\), we have

\[
\frac{\partial y}{\partial \mathbf{x}}=\begin{bmatrix}
\frac{\partial y}{\partial x_1}\\
\frac{\partial y}{\partial x_2}\\
\vdots\\
\frac{\partial y}{\partial x_n}
\end{bmatrix},\quad 
\frac{\partial \mathbf y}{\partial {x}}=\begin{bmatrix}
\frac{\partial y_1}{\partial x}, 
\frac{\partial y_2}{\partial x}, 
\ldots,
\frac{\partial y_m}{\partial x}
\end{bmatrix}
\]

and

\[
\frac{\partial \mathbf y}{\partial \mathbf{x}}
=\begin{bmatrix}
\frac{\partial \mathbf y}{\partial {x_1}}\\
\frac{\partial \mathbf y}{\partial {x_2}}\\
\vdots\\
\frac{\partial \mathbf y}{\partial {x_3}}\\
\end{bmatrix}
=\begin{bmatrix}
\frac{\partial y_1}{\partial x_1}, 
\frac{\partial y_2}{\partial x_1},
\ldots,
\frac{\partial y_m}{\partial x_1}
\\ 
\frac{\partial y_1}{\partial x_2},
\frac{\partial y_2}{\partial x_2},
\ldots,
\frac{\partial y_m}{\partial x_2}\\ 
\vdots\\
\frac{\partial y_1}{\partial x_n},
\frac{\partial y_2}{\partial x_n},
\ldots,
\frac{\partial y_m}{\partial x_n}
\end{bmatrix}
\]

Questions:

\begin{enumerate}
\def\labelenumi{\arabic{enumi}.}
\tightlist
\item
  Assume \(\mathbf y = f(\mathbf u)\) and \(\mathbf u = g(\mathbf x)\),
  write down the chain rule for
  \(\frac {\partial\mathbf y}{\partial\mathbf x}\)
\item
  Given
  \(\mathbf X \in \mathbb R^{m\times n},\ \mathbf w \in \mathbb R^n, \ \mathbf y \in \mathbb R^m\),
  assume \(z = \| \mathbf X \mathbf w - \mathbf y\|^2\), compute
  \(\frac{\partial z}{\partial\mathbf w}\).
\end{enumerate}

    \begin{enumerate}
\def\labelenumi{\arabic{enumi})}
\tightlist
\item
  For the first part we have
  \(\mathbf y=f\left(\begin{bmatrix} u_1\\ u_2\\ \vdots\\ u_n \end{bmatrix}\right)\),
  and
  \(\mathbf u=f\left(\begin{bmatrix} x_1\\ x_2\\ \vdots\\ x_n \end{bmatrix}\right)\),
  we get
  \(\mathbf y=\begin{bmatrix} f(g(x_1))\\ f(g(x_2))\\ \vdots\\ f(g(x_n)) \end{bmatrix}\)
  and according to the information we have the formula
  \(\frac{\partial \mathbf y}{\partial \mathbf{x}}\).
\end{enumerate}

So we can plug in the formula, so
\$\frac{\partial \mathbf y}{\partial \mathbf{x}} =

\begin{bmatrix}
\frac{\partial y_1}{\partial u_1}\frac{\partial u_1}{\partial x_1}, 
\frac{\partial y_1}{\partial u_2}\frac{\partial u_2}{\partial x_2},
\ldots,
\frac{\partial y_1}{\partial u_n}\frac{\partial u_n}{\partial x_n},
\\ 
\frac{\partial y_2}{\partial u_1}\frac{\partial u_1}{\partial x_1},
\frac{\partial y_2}{\partial u_2}\frac{\partial u_2}{\partial x_2},
\ldots,
\frac{\partial y_2}{\partial u_n}\frac{\partial u_n}{\partial x_n}\\ 
\vdots\\
\frac{\partial y_m}{\partial u_1}\frac{\partial u_1}{\partial x_1},
\frac{\partial y_m}{\partial u_2}\frac{\partial u_2}{\partial x_2},
\ldots,
\frac{\partial y_m}{\partial u_n}\frac{\partial u_n}{\partial x_n}
\end{bmatrix}

\$

    \begin{enumerate}
\def\labelenumi{\arabic{enumi})}
\setcounter{enumi}{1}
\tightlist
\item
  Since we have \(z = \| \mathbf X \mathbf w - \mathbf y\|^2\), we can
  get \$z = 2(\mathbf X \mathbf w - \mathbf y)\^{}T \mathbf X = 2\left(
  \textbackslash{}begin\{bmatrix\} x\_\{11\}, x\_\{12\}, \ldots,
  x\_\{1n\}\textbackslash{} x\_\{21\}, x\_\{22\}, \ldots,
  x\_\{2n\}\textbackslash{} \vdots\textbackslash{} x\_\{m1\}, x\_\{m2\},
  \ldots, x\_\{mn\}\textbackslash{} \textbackslash{}end\{bmatrix\}
  \textbackslash{}begin\{bmatrix\}
  w\_1\textbackslash{}w\_2\textbackslash{} \vdots \textbackslash{} w\_n
  \textbackslash{}end\{bmatrix\} - \textbackslash{}begin\{bmatrix\}
  y\_1\textbackslash{}y\_2\textbackslash{} \vdots \textbackslash{} y\_m
  \textbackslash{}end\{bmatrix\} \right)\^{}T
  \textbackslash{}begin\{bmatrix\} x\_\{11\}, x\_\{12\}, \ldots,
  x\_\{1n\}\textbackslash{} x\_\{21\}, x\_\{22\}, \ldots,
  x\_\{2n\}\textbackslash{} \vdots\textbackslash{} x\_\{m1\}, x\_\{m2\},
  \ldots, x\_\{mn\}\textbackslash{} \textbackslash{}end\{bmatrix\}
  \in \mathbb R\^{}m \$
\end{enumerate}

    \hypertarget{numerical-precision}{%
\subsection{4. Numerical Precision}\label{numerical-precision}}

Given scalars \texttt{x} and \texttt{y}, implement the following
\texttt{log\_exp} function such that it returns
\[-\log\left(\frac{e^x}{e^x+e^y}\right)\].

    \begin{Verbatim}[commandchars=\\\{\}]
{\color{incolor}In [{\color{incolor}103}]:} \PY{k+kn}{from} \PY{n+nn}{math} \PY{k}{import} \PY{n}{log}
          \PY{k+kn}{from} \PY{n+nn}{math} \PY{k}{import} \PY{n}{exp}
          \PY{k}{def} \PY{n+nf}{log\PYZus{}exp}\PY{p}{(}\PY{n}{x}\PY{p}{,} \PY{n}{y}\PY{p}{)}\PY{p}{:}
              \PY{c+c1}{\PYZsh{}\PYZsh{} add your solution here}
              \PY{k}{return} \PY{o}{\PYZhy{}}\PY{n}{log}\PY{p}{(}\PY{n}{exp}\PY{p}{(}\PY{n}{x}\PY{o}{.}\PY{n}{asscalar}\PY{p}{(}\PY{p}{)}\PY{p}{)}\PY{o}{/}\PY{p}{(}\PY{n}{exp}\PY{p}{(}\PY{n}{x}\PY{o}{.}\PY{n}{asscalar}\PY{p}{(}\PY{p}{)}\PY{p}{)} \PY{o}{+} \PY{n}{exp}\PY{p}{(}\PY{n}{y}\PY{o}{.}\PY{n}{asscalar}\PY{p}{(}\PY{p}{)}\PY{p}{)}\PY{p}{)}\PY{p}{)}
\end{Verbatim}


    Test your codes with normal inputs:

    \begin{Verbatim}[commandchars=\\\{\}]
{\color{incolor}In [{\color{incolor}104}]:} \PY{n}{x}\PY{p}{,} \PY{n}{y} \PY{o}{=} \PY{n}{nd}\PY{o}{.}\PY{n}{array}\PY{p}{(}\PY{p}{[}\PY{l+m+mi}{2}\PY{p}{]}\PY{p}{)}\PY{p}{,} \PY{n}{nd}\PY{o}{.}\PY{n}{array}\PY{p}{(}\PY{p}{[}\PY{l+m+mi}{3}\PY{p}{]}\PY{p}{)}
          \PY{n}{z} \PY{o}{=} \PY{n}{log\PYZus{}exp}\PY{p}{(}\PY{n}{x}\PY{p}{,} \PY{n}{y}\PY{p}{)}
          \PY{n}{z}
\end{Verbatim}


\begin{Verbatim}[commandchars=\\\{\}]
{\color{outcolor}Out[{\color{outcolor}104}]:} 1.3132616875182228
\end{Verbatim}
            
    Now implement a function to compute \(\partial z/\partial x\) and
\(\partial z/\partial y\) with \texttt{autograd}

    \begin{Verbatim}[commandchars=\\\{\}]
{\color{incolor}In [{\color{incolor}105}]:} \PY{k}{def} \PY{n+nf}{grad}\PY{p}{(}\PY{n}{forward\PYZus{}func}\PY{p}{,} \PY{n}{x}\PY{p}{,} \PY{n}{y}\PY{p}{)}\PY{p}{:} 
              \PY{c+c1}{\PYZsh{}\PYZsh{} Add your codes here}
              \PY{n+nb}{print}\PY{p}{(}\PY{l+s+s1}{\PYZsq{}}\PY{l+s+s1}{x.grad =}\PY{l+s+s1}{\PYZsq{}}\PY{p}{,} \PY{n}{x}\PY{o}{.}\PY{n}{grad}\PY{p}{)}
              \PY{n+nb}{print}\PY{p}{(}\PY{l+s+s1}{\PYZsq{}}\PY{l+s+s1}{y.grad =}\PY{l+s+s1}{\PYZsq{}}\PY{p}{,} \PY{n}{y}\PY{o}{.}\PY{n}{grad}\PY{p}{)}
\end{Verbatim}


    Test your codes, it should print the results nicely.

    \begin{Verbatim}[commandchars=\\\{\}]
{\color{incolor}In [{\color{incolor}106}]:} \PY{n}{grad}\PY{p}{(}\PY{n}{log\PYZus{}exp}\PY{p}{,} \PY{n}{x}\PY{p}{,} \PY{n}{y}\PY{p}{)}
\end{Verbatim}


    \begin{Verbatim}[commandchars=\\\{\}]
x.grad = None
y.grad = None

    \end{Verbatim}

    But now let's try some ``hard'' inputs

    \begin{Verbatim}[commandchars=\\\{\}]
{\color{incolor}In [{\color{incolor}8}]:} \PY{n}{x}\PY{p}{,} \PY{n}{y} \PY{o}{=} \PY{n}{nd}\PY{o}{.}\PY{n}{array}\PY{p}{(}\PY{p}{[}\PY{l+m+mi}{50}\PY{p}{]}\PY{p}{)}\PY{p}{,} \PY{n}{nd}\PY{o}{.}\PY{n}{array}\PY{p}{(}\PY{p}{[}\PY{l+m+mi}{100}\PY{p}{]}\PY{p}{)}
        \PY{n}{grad}\PY{p}{(}\PY{n}{log\PYZus{}exp}\PY{p}{,} \PY{n}{x}\PY{p}{,} \PY{n}{y}\PY{p}{)}
\end{Verbatim}


    \begin{Verbatim}[commandchars=\\\{\}]
x.grad = None
y.grad = None

    \end{Verbatim}

    Does your code return correct results? If not, try to understand the
reason. (Hint, evaluate \texttt{exp(100)}). Now develop a new function
\texttt{stable\_log\_exp} that is identical to \texttt{log\_exp} in
math, but returns a more numerical stable result.

    \begin{Verbatim}[commandchars=\\\{\}]
{\color{incolor}In [{\color{incolor}9}]:} \PY{k}{def} \PY{n+nf}{stable\PYZus{}log\PYZus{}exp}\PY{p}{(}\PY{n}{x}\PY{p}{,} \PY{n}{y}\PY{p}{)}\PY{p}{:}
            \PY{c+c1}{\PYZsh{}\PYZsh{} Add your codes here}
            \PY{k}{pass}
        
        \PY{n}{grad}\PY{p}{(}\PY{n}{stable\PYZus{}log\PYZus{}exp}\PY{p}{,} \PY{n}{x}\PY{p}{,} \PY{n}{y}\PY{p}{)}
\end{Verbatim}


    \begin{Verbatim}[commandchars=\\\{\}]
x.grad = None
y.grad = None

    \end{Verbatim}


    % Add a bibliography block to the postdoc
    
    
    
    \end{document}
