
% Default to the notebook output style

    


% Inherit from the specified cell style.




    
\documentclass[11pt]{article}

    
    
    \usepackage[T1]{fontenc}
    % Nicer default font (+ math font) than Computer Modern for most use cases
    \usepackage{mathpazo}

    % Basic figure setup, for now with no caption control since it's done
    % automatically by Pandoc (which extracts ![](path) syntax from Markdown).
    \usepackage{graphicx}
    % We will generate all images so they have a width \maxwidth. This means
    % that they will get their normal width if they fit onto the page, but
    % are scaled down if they would overflow the margins.
    \makeatletter
    \def\maxwidth{\ifdim\Gin@nat@width>\linewidth\linewidth
    \else\Gin@nat@width\fi}
    \makeatother
    \let\Oldincludegraphics\includegraphics
    % Set max figure width to be 80% of text width, for now hardcoded.
    \renewcommand{\includegraphics}[1]{\Oldincludegraphics[width=.8\maxwidth]{#1}}
    % Ensure that by default, figures have no caption (until we provide a
    % proper Figure object with a Caption API and a way to capture that
    % in the conversion process - todo).
    \usepackage{caption}
    \DeclareCaptionLabelFormat{nolabel}{}
    \captionsetup{labelformat=nolabel}

    \usepackage{adjustbox} % Used to constrain images to a maximum size 
    \usepackage{xcolor} % Allow colors to be defined
    \usepackage{enumerate} % Needed for markdown enumerations to work
    \usepackage{geometry} % Used to adjust the document margins
    \usepackage{amsmath} % Equations
    \usepackage{amssymb} % Equations
    \usepackage{textcomp} % defines textquotesingle
    % Hack from http://tex.stackexchange.com/a/47451/13684:
    \AtBeginDocument{%
        \def\PYZsq{\textquotesingle}% Upright quotes in Pygmentized code
    }
    \usepackage{upquote} % Upright quotes for verbatim code
    \usepackage{eurosym} % defines \euro
    \usepackage[mathletters]{ucs} % Extended unicode (utf-8) support
    \usepackage[utf8x]{inputenc} % Allow utf-8 characters in the tex document
    \usepackage{fancyvrb} % verbatim replacement that allows latex
    \usepackage{grffile} % extends the file name processing of package graphics 
                         % to support a larger range 
    % The hyperref package gives us a pdf with properly built
    % internal navigation ('pdf bookmarks' for the table of contents,
    % internal cross-reference links, web links for URLs, etc.)
    \usepackage{hyperref}
    \usepackage{longtable} % longtable support required by pandoc >1.10
    \usepackage{booktabs}  % table support for pandoc > 1.12.2
    \usepackage[inline]{enumitem} % IRkernel/repr support (it uses the enumerate* environment)
    \usepackage[normalem]{ulem} % ulem is needed to support strikethroughs (\sout)
                                % normalem makes italics be italics, not underlines
    

    
    
    % Colors for the hyperref package
    \definecolor{urlcolor}{rgb}{0,.145,.698}
    \definecolor{linkcolor}{rgb}{.71,0.21,0.01}
    \definecolor{citecolor}{rgb}{.12,.54,.11}

    % ANSI colors
    \definecolor{ansi-black}{HTML}{3E424D}
    \definecolor{ansi-black-intense}{HTML}{282C36}
    \definecolor{ansi-red}{HTML}{E75C58}
    \definecolor{ansi-red-intense}{HTML}{B22B31}
    \definecolor{ansi-green}{HTML}{00A250}
    \definecolor{ansi-green-intense}{HTML}{007427}
    \definecolor{ansi-yellow}{HTML}{DDB62B}
    \definecolor{ansi-yellow-intense}{HTML}{B27D12}
    \definecolor{ansi-blue}{HTML}{208FFB}
    \definecolor{ansi-blue-intense}{HTML}{0065CA}
    \definecolor{ansi-magenta}{HTML}{D160C4}
    \definecolor{ansi-magenta-intense}{HTML}{A03196}
    \definecolor{ansi-cyan}{HTML}{60C6C8}
    \definecolor{ansi-cyan-intense}{HTML}{258F8F}
    \definecolor{ansi-white}{HTML}{C5C1B4}
    \definecolor{ansi-white-intense}{HTML}{A1A6B2}

    % commands and environments needed by pandoc snippets
    % extracted from the output of `pandoc -s`
    \providecommand{\tightlist}{%
      \setlength{\itemsep}{0pt}\setlength{\parskip}{0pt}}
    \DefineVerbatimEnvironment{Highlighting}{Verbatim}{commandchars=\\\{\}}
    % Add ',fontsize=\small' for more characters per line
    \newenvironment{Shaded}{}{}
    \newcommand{\KeywordTok}[1]{\textcolor[rgb]{0.00,0.44,0.13}{\textbf{{#1}}}}
    \newcommand{\DataTypeTok}[1]{\textcolor[rgb]{0.56,0.13,0.00}{{#1}}}
    \newcommand{\DecValTok}[1]{\textcolor[rgb]{0.25,0.63,0.44}{{#1}}}
    \newcommand{\BaseNTok}[1]{\textcolor[rgb]{0.25,0.63,0.44}{{#1}}}
    \newcommand{\FloatTok}[1]{\textcolor[rgb]{0.25,0.63,0.44}{{#1}}}
    \newcommand{\CharTok}[1]{\textcolor[rgb]{0.25,0.44,0.63}{{#1}}}
    \newcommand{\StringTok}[1]{\textcolor[rgb]{0.25,0.44,0.63}{{#1}}}
    \newcommand{\CommentTok}[1]{\textcolor[rgb]{0.38,0.63,0.69}{\textit{{#1}}}}
    \newcommand{\OtherTok}[1]{\textcolor[rgb]{0.00,0.44,0.13}{{#1}}}
    \newcommand{\AlertTok}[1]{\textcolor[rgb]{1.00,0.00,0.00}{\textbf{{#1}}}}
    \newcommand{\FunctionTok}[1]{\textcolor[rgb]{0.02,0.16,0.49}{{#1}}}
    \newcommand{\RegionMarkerTok}[1]{{#1}}
    \newcommand{\ErrorTok}[1]{\textcolor[rgb]{1.00,0.00,0.00}{\textbf{{#1}}}}
    \newcommand{\NormalTok}[1]{{#1}}
    
    % Additional commands for more recent versions of Pandoc
    \newcommand{\ConstantTok}[1]{\textcolor[rgb]{0.53,0.00,0.00}{{#1}}}
    \newcommand{\SpecialCharTok}[1]{\textcolor[rgb]{0.25,0.44,0.63}{{#1}}}
    \newcommand{\VerbatimStringTok}[1]{\textcolor[rgb]{0.25,0.44,0.63}{{#1}}}
    \newcommand{\SpecialStringTok}[1]{\textcolor[rgb]{0.73,0.40,0.53}{{#1}}}
    \newcommand{\ImportTok}[1]{{#1}}
    \newcommand{\DocumentationTok}[1]{\textcolor[rgb]{0.73,0.13,0.13}{\textit{{#1}}}}
    \newcommand{\AnnotationTok}[1]{\textcolor[rgb]{0.38,0.63,0.69}{\textbf{\textit{{#1}}}}}
    \newcommand{\CommentVarTok}[1]{\textcolor[rgb]{0.38,0.63,0.69}{\textbf{\textit{{#1}}}}}
    \newcommand{\VariableTok}[1]{\textcolor[rgb]{0.10,0.09,0.49}{{#1}}}
    \newcommand{\ControlFlowTok}[1]{\textcolor[rgb]{0.00,0.44,0.13}{\textbf{{#1}}}}
    \newcommand{\OperatorTok}[1]{\textcolor[rgb]{0.40,0.40,0.40}{{#1}}}
    \newcommand{\BuiltInTok}[1]{{#1}}
    \newcommand{\ExtensionTok}[1]{{#1}}
    \newcommand{\PreprocessorTok}[1]{\textcolor[rgb]{0.74,0.48,0.00}{{#1}}}
    \newcommand{\AttributeTok}[1]{\textcolor[rgb]{0.49,0.56,0.16}{{#1}}}
    \newcommand{\InformationTok}[1]{\textcolor[rgb]{0.38,0.63,0.69}{\textbf{\textit{{#1}}}}}
    \newcommand{\WarningTok}[1]{\textcolor[rgb]{0.38,0.63,0.69}{\textbf{\textit{{#1}}}}}
    
    
    % Define a nice break command that doesn't care if a line doesn't already
    % exist.
    \def\br{\hspace*{\fill} \\* }
    % Math Jax compatability definitions
    \def\gt{>}
    \def\lt{<}
    % Document parameters
    \title{homework3}
    
    
    

    % Pygments definitions
    
\makeatletter
\def\PY@reset{\let\PY@it=\relax \let\PY@bf=\relax%
    \let\PY@ul=\relax \let\PY@tc=\relax%
    \let\PY@bc=\relax \let\PY@ff=\relax}
\def\PY@tok#1{\csname PY@tok@#1\endcsname}
\def\PY@toks#1+{\ifx\relax#1\empty\else%
    \PY@tok{#1}\expandafter\PY@toks\fi}
\def\PY@do#1{\PY@bc{\PY@tc{\PY@ul{%
    \PY@it{\PY@bf{\PY@ff{#1}}}}}}}
\def\PY#1#2{\PY@reset\PY@toks#1+\relax+\PY@do{#2}}

\expandafter\def\csname PY@tok@w\endcsname{\def\PY@tc##1{\textcolor[rgb]{0.73,0.73,0.73}{##1}}}
\expandafter\def\csname PY@tok@c\endcsname{\let\PY@it=\textit\def\PY@tc##1{\textcolor[rgb]{0.25,0.50,0.50}{##1}}}
\expandafter\def\csname PY@tok@cp\endcsname{\def\PY@tc##1{\textcolor[rgb]{0.74,0.48,0.00}{##1}}}
\expandafter\def\csname PY@tok@k\endcsname{\let\PY@bf=\textbf\def\PY@tc##1{\textcolor[rgb]{0.00,0.50,0.00}{##1}}}
\expandafter\def\csname PY@tok@kp\endcsname{\def\PY@tc##1{\textcolor[rgb]{0.00,0.50,0.00}{##1}}}
\expandafter\def\csname PY@tok@kt\endcsname{\def\PY@tc##1{\textcolor[rgb]{0.69,0.00,0.25}{##1}}}
\expandafter\def\csname PY@tok@o\endcsname{\def\PY@tc##1{\textcolor[rgb]{0.40,0.40,0.40}{##1}}}
\expandafter\def\csname PY@tok@ow\endcsname{\let\PY@bf=\textbf\def\PY@tc##1{\textcolor[rgb]{0.67,0.13,1.00}{##1}}}
\expandafter\def\csname PY@tok@nb\endcsname{\def\PY@tc##1{\textcolor[rgb]{0.00,0.50,0.00}{##1}}}
\expandafter\def\csname PY@tok@nf\endcsname{\def\PY@tc##1{\textcolor[rgb]{0.00,0.00,1.00}{##1}}}
\expandafter\def\csname PY@tok@nc\endcsname{\let\PY@bf=\textbf\def\PY@tc##1{\textcolor[rgb]{0.00,0.00,1.00}{##1}}}
\expandafter\def\csname PY@tok@nn\endcsname{\let\PY@bf=\textbf\def\PY@tc##1{\textcolor[rgb]{0.00,0.00,1.00}{##1}}}
\expandafter\def\csname PY@tok@ne\endcsname{\let\PY@bf=\textbf\def\PY@tc##1{\textcolor[rgb]{0.82,0.25,0.23}{##1}}}
\expandafter\def\csname PY@tok@nv\endcsname{\def\PY@tc##1{\textcolor[rgb]{0.10,0.09,0.49}{##1}}}
\expandafter\def\csname PY@tok@no\endcsname{\def\PY@tc##1{\textcolor[rgb]{0.53,0.00,0.00}{##1}}}
\expandafter\def\csname PY@tok@nl\endcsname{\def\PY@tc##1{\textcolor[rgb]{0.63,0.63,0.00}{##1}}}
\expandafter\def\csname PY@tok@ni\endcsname{\let\PY@bf=\textbf\def\PY@tc##1{\textcolor[rgb]{0.60,0.60,0.60}{##1}}}
\expandafter\def\csname PY@tok@na\endcsname{\def\PY@tc##1{\textcolor[rgb]{0.49,0.56,0.16}{##1}}}
\expandafter\def\csname PY@tok@nt\endcsname{\let\PY@bf=\textbf\def\PY@tc##1{\textcolor[rgb]{0.00,0.50,0.00}{##1}}}
\expandafter\def\csname PY@tok@nd\endcsname{\def\PY@tc##1{\textcolor[rgb]{0.67,0.13,1.00}{##1}}}
\expandafter\def\csname PY@tok@s\endcsname{\def\PY@tc##1{\textcolor[rgb]{0.73,0.13,0.13}{##1}}}
\expandafter\def\csname PY@tok@sd\endcsname{\let\PY@it=\textit\def\PY@tc##1{\textcolor[rgb]{0.73,0.13,0.13}{##1}}}
\expandafter\def\csname PY@tok@si\endcsname{\let\PY@bf=\textbf\def\PY@tc##1{\textcolor[rgb]{0.73,0.40,0.53}{##1}}}
\expandafter\def\csname PY@tok@se\endcsname{\let\PY@bf=\textbf\def\PY@tc##1{\textcolor[rgb]{0.73,0.40,0.13}{##1}}}
\expandafter\def\csname PY@tok@sr\endcsname{\def\PY@tc##1{\textcolor[rgb]{0.73,0.40,0.53}{##1}}}
\expandafter\def\csname PY@tok@ss\endcsname{\def\PY@tc##1{\textcolor[rgb]{0.10,0.09,0.49}{##1}}}
\expandafter\def\csname PY@tok@sx\endcsname{\def\PY@tc##1{\textcolor[rgb]{0.00,0.50,0.00}{##1}}}
\expandafter\def\csname PY@tok@m\endcsname{\def\PY@tc##1{\textcolor[rgb]{0.40,0.40,0.40}{##1}}}
\expandafter\def\csname PY@tok@gh\endcsname{\let\PY@bf=\textbf\def\PY@tc##1{\textcolor[rgb]{0.00,0.00,0.50}{##1}}}
\expandafter\def\csname PY@tok@gu\endcsname{\let\PY@bf=\textbf\def\PY@tc##1{\textcolor[rgb]{0.50,0.00,0.50}{##1}}}
\expandafter\def\csname PY@tok@gd\endcsname{\def\PY@tc##1{\textcolor[rgb]{0.63,0.00,0.00}{##1}}}
\expandafter\def\csname PY@tok@gi\endcsname{\def\PY@tc##1{\textcolor[rgb]{0.00,0.63,0.00}{##1}}}
\expandafter\def\csname PY@tok@gr\endcsname{\def\PY@tc##1{\textcolor[rgb]{1.00,0.00,0.00}{##1}}}
\expandafter\def\csname PY@tok@ge\endcsname{\let\PY@it=\textit}
\expandafter\def\csname PY@tok@gs\endcsname{\let\PY@bf=\textbf}
\expandafter\def\csname PY@tok@gp\endcsname{\let\PY@bf=\textbf\def\PY@tc##1{\textcolor[rgb]{0.00,0.00,0.50}{##1}}}
\expandafter\def\csname PY@tok@go\endcsname{\def\PY@tc##1{\textcolor[rgb]{0.53,0.53,0.53}{##1}}}
\expandafter\def\csname PY@tok@gt\endcsname{\def\PY@tc##1{\textcolor[rgb]{0.00,0.27,0.87}{##1}}}
\expandafter\def\csname PY@tok@err\endcsname{\def\PY@bc##1{\setlength{\fboxsep}{0pt}\fcolorbox[rgb]{1.00,0.00,0.00}{1,1,1}{\strut ##1}}}
\expandafter\def\csname PY@tok@kc\endcsname{\let\PY@bf=\textbf\def\PY@tc##1{\textcolor[rgb]{0.00,0.50,0.00}{##1}}}
\expandafter\def\csname PY@tok@kd\endcsname{\let\PY@bf=\textbf\def\PY@tc##1{\textcolor[rgb]{0.00,0.50,0.00}{##1}}}
\expandafter\def\csname PY@tok@kn\endcsname{\let\PY@bf=\textbf\def\PY@tc##1{\textcolor[rgb]{0.00,0.50,0.00}{##1}}}
\expandafter\def\csname PY@tok@kr\endcsname{\let\PY@bf=\textbf\def\PY@tc##1{\textcolor[rgb]{0.00,0.50,0.00}{##1}}}
\expandafter\def\csname PY@tok@bp\endcsname{\def\PY@tc##1{\textcolor[rgb]{0.00,0.50,0.00}{##1}}}
\expandafter\def\csname PY@tok@fm\endcsname{\def\PY@tc##1{\textcolor[rgb]{0.00,0.00,1.00}{##1}}}
\expandafter\def\csname PY@tok@vc\endcsname{\def\PY@tc##1{\textcolor[rgb]{0.10,0.09,0.49}{##1}}}
\expandafter\def\csname PY@tok@vg\endcsname{\def\PY@tc##1{\textcolor[rgb]{0.10,0.09,0.49}{##1}}}
\expandafter\def\csname PY@tok@vi\endcsname{\def\PY@tc##1{\textcolor[rgb]{0.10,0.09,0.49}{##1}}}
\expandafter\def\csname PY@tok@vm\endcsname{\def\PY@tc##1{\textcolor[rgb]{0.10,0.09,0.49}{##1}}}
\expandafter\def\csname PY@tok@sa\endcsname{\def\PY@tc##1{\textcolor[rgb]{0.73,0.13,0.13}{##1}}}
\expandafter\def\csname PY@tok@sb\endcsname{\def\PY@tc##1{\textcolor[rgb]{0.73,0.13,0.13}{##1}}}
\expandafter\def\csname PY@tok@sc\endcsname{\def\PY@tc##1{\textcolor[rgb]{0.73,0.13,0.13}{##1}}}
\expandafter\def\csname PY@tok@dl\endcsname{\def\PY@tc##1{\textcolor[rgb]{0.73,0.13,0.13}{##1}}}
\expandafter\def\csname PY@tok@s2\endcsname{\def\PY@tc##1{\textcolor[rgb]{0.73,0.13,0.13}{##1}}}
\expandafter\def\csname PY@tok@sh\endcsname{\def\PY@tc##1{\textcolor[rgb]{0.73,0.13,0.13}{##1}}}
\expandafter\def\csname PY@tok@s1\endcsname{\def\PY@tc##1{\textcolor[rgb]{0.73,0.13,0.13}{##1}}}
\expandafter\def\csname PY@tok@mb\endcsname{\def\PY@tc##1{\textcolor[rgb]{0.40,0.40,0.40}{##1}}}
\expandafter\def\csname PY@tok@mf\endcsname{\def\PY@tc##1{\textcolor[rgb]{0.40,0.40,0.40}{##1}}}
\expandafter\def\csname PY@tok@mh\endcsname{\def\PY@tc##1{\textcolor[rgb]{0.40,0.40,0.40}{##1}}}
\expandafter\def\csname PY@tok@mi\endcsname{\def\PY@tc##1{\textcolor[rgb]{0.40,0.40,0.40}{##1}}}
\expandafter\def\csname PY@tok@il\endcsname{\def\PY@tc##1{\textcolor[rgb]{0.40,0.40,0.40}{##1}}}
\expandafter\def\csname PY@tok@mo\endcsname{\def\PY@tc##1{\textcolor[rgb]{0.40,0.40,0.40}{##1}}}
\expandafter\def\csname PY@tok@ch\endcsname{\let\PY@it=\textit\def\PY@tc##1{\textcolor[rgb]{0.25,0.50,0.50}{##1}}}
\expandafter\def\csname PY@tok@cm\endcsname{\let\PY@it=\textit\def\PY@tc##1{\textcolor[rgb]{0.25,0.50,0.50}{##1}}}
\expandafter\def\csname PY@tok@cpf\endcsname{\let\PY@it=\textit\def\PY@tc##1{\textcolor[rgb]{0.25,0.50,0.50}{##1}}}
\expandafter\def\csname PY@tok@c1\endcsname{\let\PY@it=\textit\def\PY@tc##1{\textcolor[rgb]{0.25,0.50,0.50}{##1}}}
\expandafter\def\csname PY@tok@cs\endcsname{\let\PY@it=\textit\def\PY@tc##1{\textcolor[rgb]{0.25,0.50,0.50}{##1}}}

\def\PYZbs{\char`\\}
\def\PYZus{\char`\_}
\def\PYZob{\char`\{}
\def\PYZcb{\char`\}}
\def\PYZca{\char`\^}
\def\PYZam{\char`\&}
\def\PYZlt{\char`\<}
\def\PYZgt{\char`\>}
\def\PYZsh{\char`\#}
\def\PYZpc{\char`\%}
\def\PYZdl{\char`\$}
\def\PYZhy{\char`\-}
\def\PYZsq{\char`\'}
\def\PYZdq{\char`\"}
\def\PYZti{\char`\~}
% for compatibility with earlier versions
\def\PYZat{@}
\def\PYZlb{[}
\def\PYZrb{]}
\makeatother


    % Exact colors from NB
    \definecolor{incolor}{rgb}{0.0, 0.0, 0.5}
    \definecolor{outcolor}{rgb}{0.545, 0.0, 0.0}



    
    % Prevent overflowing lines due to hard-to-break entities
    \sloppy 
    % Setup hyperref package
    \hypersetup{
      breaklinks=true,  % so long urls are correctly broken across lines
      colorlinks=true,
      urlcolor=urlcolor,
      linkcolor=linkcolor,
      citecolor=citecolor,
      }
    % Slightly bigger margins than the latex defaults
    
    \geometry{verbose,tmargin=1in,bmargin=1in,lmargin=1in,rmargin=1in}
    
    

    \begin{document}
    
    
    \maketitle
    
    

    
    \hypertarget{homework-3---berkeley-stat-157}{%
\section{Homework 3 - Berkeley STAT
157}\label{homework-3---berkeley-stat-157}}

Handout 2/5/2019, due 2/12/2019 by 4pm in Git by committing to your
repository.

\textbf{Formatting: please include both a .ipynb and .pdf file in your
homework submission, named homework3.ipynb and homework3.pdf. You can
export your notebook to a pdf either by File -\textgreater{} Download as
-\textgreater{} PDF via Latex (you may need Latex installed), or by
simply printing to a pdf from your browser (you may want to do File
-\textgreater{} Print Preview in jupyter first). Please don't change the
filename.}

    \begin{Verbatim}[commandchars=\\\{\}]
{\color{incolor}In [{\color{incolor}1}]:} \PY{k+kn}{from} \PY{n+nn}{mxnet} \PY{k}{import} \PY{n}{nd}\PY{p}{,} \PY{n}{autograd}\PY{p}{,} \PY{n}{gluon}
        \PY{k+kn}{import} \PY{n+nn}{matplotlib}\PY{n+nn}{.}\PY{n+nn}{pyplot} \PY{k}{as} \PY{n+nn}{plt}
\end{Verbatim}


    \hypertarget{logistic-regression-for-binary-classification}{%
\section{1. Logistic Regression for Binary
Classification}\label{logistic-regression-for-binary-classification}}

In multiclass classification we typically use the exponential model

\[p(y|\mathbf{o}) = \mathrm{softmax}(\mathbf{o})_y = \frac{\exp(o_y)}{\sum_{y'} \exp(o_{y'})}\]

1.1. Show that this parametrization has a spurious degree of freedom.
That is, show that both \(\mathbf{o}\) and \(\mathbf{o} + c\) with
\(c \in \mathbb{R}\) lead to the same probability estimate. 1.2. For
binary classification, i.e.~whenever we have only two classes
\(\{-1, 1\}\), we can arbitrarily set \(o_{-1} = 0\). Using the
shorthand \(o = o_1\) show that this is equivalent to

\[p(y=1|o) = \frac{1}{1 + \exp(-o)}\]

1.3. Show that the log-likelihood loss (often called logistic loss) for
labels \(y \in \{-1, 1\}\) is thus given by

\[-\log p(y|o) = \log (1 + \exp(-y \cdot o))\]

1.4. Show that for \(y = 1\) the logistic loss asymptotes to \(o\) for
\(o \to \infty\) and to \(\exp(o)\) for \(o \to -\infty\).

    \hypertarget{section}{%
\subsubsection{1.1}\label{section}}

Since we have
\[p(y|\mathbf{o}) = \mathrm{softmax}(\mathbf{o})_y = \frac{\exp(o_y)}{\sum_{y'} \exp(o_{y'})}\]
we can use this result to add a constant \(c\) on every term, that is
\(p(y|\mathbf{o+c}) = \mathrm{softmax}(\mathbf{o+c})_y = \frac{\exp(o_y + c)}{\sum_{y'} \exp(o_{y' + c})} = \frac{\exp(o_y)\exp(c)}{\sum_{y'} \exp(o_{y'})\exp(c)} = \frac{\exp(o_y)\exp(c)}{\exp(c)\sum_{y'} \exp(o_{y'})} = \frac{\exp(o_y)}{\sum_{y'} \exp(o_{y'})} = p(y|\mathbf{o})\).

    \hypertarget{section}{%
\subsubsection{1.2}\label{section}}

    \hypertarget{logistic-regression-and-autograd}{%
\section{2. Logistic Regression and
Autograd}\label{logistic-regression-and-autograd}}

\begin{enumerate}
\def\labelenumi{\arabic{enumi}.}
\tightlist
\item
  Implement the binary logistic loss
  \(l(y,o) = \log (1 + \exp(-y \cdot o))\) in Gluon
\item
  Plot its values for \(y \in \{-1, 1\}\) over the range of
  \(o \in [-5, 5]\).
\item
  Plot its derivative with respect to \(o\) for \(o \in [-5, 5]\) using
  `autograd'.
\end{enumerate}

    \begin{Verbatim}[commandchars=\\\{\}]
{\color{incolor}In [{\color{incolor}2}]:} \PY{k}{def} \PY{n+nf}{loss}\PY{p}{(}\PY{n}{y}\PY{p}{,}\PY{n}{o}\PY{p}{)}\PY{p}{:}
            \PY{c+c1}{\PYZsh{}\PYZsh{} add your loss function here}
            \PY{k}{return} \PY{n}{l}
\end{Verbatim}


    \hypertarget{ohms-law}{%
\section{3. Ohm's Law}\label{ohms-law}}

Imagine that you're a young physicist, maybe named
\href{https://en.wikipedia.org/wiki/Georg_Ohm}{Georg Simon Ohm}, trying
to figure out how current and voltage depend on each other for
resistors. You have some idea but you aren't quite sure yet whether the
dependence is linear or quadratic. So you take some measurements,
conveniently given to you as `ndarrays' in Python. They are indicated by
`current' and `voltage'.

Your goal is to use least mean squares regression to identify the
coefficients for the following three models using automatic
differentiation and least mean squares regression. The three models are:

\begin{enumerate}
\def\labelenumi{\arabic{enumi}.}
\tightlist
\item
  Quadratic model where
  \(\mathrm{voltage} = c + r \cdot \mathrm{current} + q \cdot \mathrm{current}^2\).
\item
  Linear model where
  \(\mathrm{voltage} = c + r \cdot \mathrm{current}\).
\item
  Ohm's law where \(\mathrm{voltage} = r \cdot \mathrm{current}\).
\end{enumerate}

    \begin{Verbatim}[commandchars=\\\{\}]
{\color{incolor}In [{\color{incolor}3}]:} \PY{n}{current} \PY{o}{=} \PY{n}{nd}\PY{o}{.}\PY{n}{array}\PY{p}{(}\PY{p}{[}\PY{l+m+mf}{1.5420291}\PY{p}{,} \PY{l+m+mf}{1.8935232}\PY{p}{,} \PY{l+m+mf}{2.1603365}\PY{p}{,} \PY{l+m+mf}{2.5381863}\PY{p}{,} \PY{l+m+mf}{2.893443}\PY{p}{,} \PYZbs{}
                            \PY{l+m+mf}{3.838855}\PY{p}{,} \PY{l+m+mf}{3.925425}\PY{p}{,} \PY{l+m+mf}{4.2233696}\PY{p}{,} \PY{l+m+mf}{4.235571}\PY{p}{,} \PY{l+m+mf}{4.273397}\PY{p}{,} \PYZbs{}
                            \PY{l+m+mf}{4.9332876}\PY{p}{,} \PY{l+m+mf}{6.4704757}\PY{p}{,} \PY{l+m+mf}{6.517571}\PY{p}{,} \PY{l+m+mf}{6.87826}\PY{p}{,} \PY{l+m+mf}{7.0009003}\PY{p}{,} \PYZbs{}
                            \PY{l+m+mf}{7.035741}\PY{p}{,} \PY{l+m+mf}{7.278681}\PY{p}{,} \PY{l+m+mf}{7.7561755}\PY{p}{,} \PY{l+m+mf}{9.121138}\PY{p}{,} \PY{l+m+mf}{9.728281}\PY{p}{]}\PY{p}{)}
        \PY{n}{voltage} \PY{o}{=} \PY{n}{nd}\PY{o}{.}\PY{n}{array}\PY{p}{(}\PY{p}{[}\PY{l+m+mf}{63.802246}\PY{p}{,} \PY{l+m+mf}{80.036026}\PY{p}{,} \PY{l+m+mf}{91.4903}\PY{p}{,} \PY{l+m+mf}{108.28776}\PY{p}{,} \PY{l+m+mf}{122.781975}\PY{p}{,} \PYZbs{}
                            \PY{l+m+mf}{161.36314}\PY{p}{,} \PY{l+m+mf}{166.50816}\PY{p}{,} \PY{l+m+mf}{176.16772}\PY{p}{,} \PY{l+m+mf}{180.29395}\PY{p}{,} \PY{l+m+mf}{179.09758}\PY{p}{,} \PYZbs{}
                            \PY{l+m+mf}{206.21027}\PY{p}{,} \PY{l+m+mf}{272.71857}\PY{p}{,} \PY{l+m+mf}{272.24033}\PY{p}{,} \PY{l+m+mf}{289.54745}\PY{p}{,} \PY{l+m+mf}{293.8488}\PY{p}{,} \PYZbs{}
                            \PY{l+m+mf}{295.2281}\PY{p}{,} \PY{l+m+mf}{306.62274}\PY{p}{,} \PY{l+m+mf}{327.93243}\PY{p}{,} \PY{l+m+mf}{383.16296}\PY{p}{,} \PY{l+m+mf}{408.65967}\PY{p}{]}\PY{p}{)}
\end{Verbatim}


    \hypertarget{entropy}{%
\section{4. Entropy}\label{entropy}}

Let's compute the \emph{binary} entropy of a number of interesting data
sources.

\begin{enumerate}
\def\labelenumi{\arabic{enumi}.}
\tightlist
\item
  Assume that you're watching the output generated by a
  \href{https://en.wikipedia.org/wiki/File:Chimpanzee_seated_at_typewriter.jpg}{monkey
  at a typewriter}. The monkey presses any of the \(44\) keys of the
  typewriter at random (you can assume that it has not discovered any
  special keys or the shift key yet). How many bits of randomness per
  character do you observe?
\item
  Unhappy with the monkey you replaced it by a drunk typesetter. It is
  able to generate words, albeit not coherently. Instead, it picks a
  random word out of a vocabulary of \(2,000\) words. Moreover, assume
  that the average length of a word is \(4.5\) letters in English. How
  many bits of randomness do you observe now?
\item
  Still unhappy with the result you replace the typesetter by a high
  quality language model. These can obtain perplexity numbers as low as
  20 points per character. The perplexity is defined as a length
  normalized probability, i.e.
\end{enumerate}

\[\mathrm{PPL}(x) = \left[p(x)\right]^{1/\mathrm{length}(x)}\]

    \hypertarget{wiens-approximation-for-the-temperature-bonus}{%
\section{5. Wien's Approximation for the Temperature
(bonus)}\label{wiens-approximation-for-the-temperature-bonus}}

We will now abuse Gluon to estimate the temperature of a black body. The
energy emanated from a black body is given by Wien's approximation.

\[B_\lambda(T) = \frac{2 h c^2}{\lambda^5} \exp\left(-\frac{h c}{\lambda k T}\right)\]

That is, the amount of energy depends on the fifth power of the
wavelength \(\lambda\) and the temperature \(T\) of the body. The latter
ensures a cutoff beyond a temperature-characteristic peak. Let us define
this and plot it.

    \begin{Verbatim}[commandchars=\\\{\}]
{\color{incolor}In [{\color{incolor}4}]:} \PY{c+c1}{\PYZsh{} Lightspeed}
        \PY{n}{c} \PY{o}{=} \PY{l+m+mi}{299792458}
        \PY{c+c1}{\PYZsh{} Planck\PYZsq{}s constant}
        \PY{n}{h} \PY{o}{=} \PY{l+m+mf}{6.62607004e\PYZhy{}34}
        \PY{c+c1}{\PYZsh{} Boltzmann constant}
        \PY{n}{k} \PY{o}{=} \PY{l+m+mf}{1.38064852e\PYZhy{}23}
        \PY{c+c1}{\PYZsh{} Wavelength scale (nanometers)}
        \PY{n}{lamscale} \PY{o}{=} \PY{l+m+mf}{1e\PYZhy{}6}
        \PY{c+c1}{\PYZsh{} Pulling out all powers of 10 upfront}
        \PY{n}{p\PYZus{}out} \PY{o}{=} \PY{l+m+mi}{2} \PY{o}{*} \PY{n}{h} \PY{o}{*} \PY{n}{c}\PY{o}{*}\PY{o}{*}\PY{l+m+mi}{2} \PY{o}{/} \PY{n}{lamscale}\PY{o}{*}\PY{o}{*}\PY{l+m+mi}{5}
        \PY{n}{p\PYZus{}in} \PY{o}{=} \PY{p}{(}\PY{n}{h} \PY{o}{/} \PY{n}{k}\PY{p}{)} \PY{o}{*} \PY{p}{(}\PY{n}{c}\PY{o}{/}\PY{n}{lamscale}\PY{p}{)} 
        
        \PY{c+c1}{\PYZsh{} Wien\PYZsq{}s law}
        \PY{k}{def} \PY{n+nf}{wien}\PY{p}{(}\PY{n}{lam}\PY{p}{,} \PY{n}{t}\PY{p}{)}\PY{p}{:}
            \PY{k}{return} \PY{p}{(}\PY{n}{p\PYZus{}out} \PY{o}{/} \PY{n}{lam}\PY{o}{*}\PY{o}{*}\PY{l+m+mi}{5}\PY{p}{)} \PY{o}{*} \PY{n}{nd}\PY{o}{.}\PY{n}{exp}\PY{p}{(}\PY{o}{\PYZhy{}}\PY{n}{p\PYZus{}in} \PY{o}{/} \PY{p}{(}\PY{n}{lam} \PY{o}{*} \PY{n}{t}\PY{p}{)}\PY{p}{)}
        
        \PY{c+c1}{\PYZsh{} Plot the radiance for a few different temperatures}
        \PY{n}{lam} \PY{o}{=} \PY{n}{nd}\PY{o}{.}\PY{n}{arange}\PY{p}{(}\PY{l+m+mi}{0}\PY{p}{,}\PY{l+m+mi}{100}\PY{p}{,}\PY{l+m+mf}{0.01}\PY{p}{)}
        \PY{k}{for} \PY{n}{t} \PY{o+ow}{in} \PY{p}{[}\PY{l+m+mi}{10}\PY{p}{,} \PY{l+m+mi}{100}\PY{p}{,} \PY{l+m+mi}{150}\PY{p}{,} \PY{l+m+mi}{200}\PY{p}{,} \PY{l+m+mi}{250}\PY{p}{,} \PY{l+m+mi}{300}\PY{p}{,} \PY{l+m+mi}{350}\PY{p}{]}\PY{p}{:}
            \PY{n}{radiance} \PY{o}{=} \PY{n}{wien}\PY{p}{(}\PY{n}{lam}\PY{p}{,} \PY{n}{t}\PY{p}{)}
            \PY{n}{plt}\PY{o}{.}\PY{n}{plot}\PY{p}{(}\PY{n}{lam}\PY{o}{.}\PY{n}{asnumpy}\PY{p}{(}\PY{p}{)}\PY{p}{,} \PY{n}{radiance}\PY{o}{.}\PY{n}{asnumpy}\PY{p}{(}\PY{p}{)}\PY{p}{,} \PY{n}{label}\PY{o}{=}\PY{p}{(}\PY{l+s+s1}{\PYZsq{}}\PY{l+s+s1}{T=}\PY{l+s+s1}{\PYZsq{}} \PY{o}{+} \PY{n+nb}{str}\PY{p}{(}\PY{n}{t}\PY{p}{)} \PY{o}{+} \PY{l+s+s1}{\PYZsq{}}\PY{l+s+s1}{K}\PY{l+s+s1}{\PYZsq{}}\PY{p}{)}\PY{p}{)}
        \PY{n}{plt}\PY{o}{.}\PY{n}{legend}\PY{p}{(}\PY{p}{)}
        \PY{n}{plt}\PY{o}{.}\PY{n}{show}\PY{p}{(}\PY{p}{)}
\end{Verbatim}


    \begin{center}
    \adjustimage{max size={0.9\linewidth}{0.9\paperheight}}{output_11_0.png}
    \end{center}
    { \hspace*{\fill} \\}
    
    Next we assume that we are a fearless physicist measuring some data. Of
course, we need to pretend that we don't really know the temperature.
But we measure the radiation at a few wavelengths.

    \begin{Verbatim}[commandchars=\\\{\}]
{\color{incolor}In [{\color{incolor}5}]:} \PY{c+c1}{\PYZsh{} real temperature is approximately 0C}
        \PY{n}{realtemp} \PY{o}{=} \PY{l+m+mi}{273}
        \PY{c+c1}{\PYZsh{} we observe at 3000nm up to 20,000nm wavelength}
        \PY{n}{wavelengths} \PY{o}{=} \PY{n}{nd}\PY{o}{.}\PY{n}{arange}\PY{p}{(}\PY{l+m+mi}{3}\PY{p}{,}\PY{l+m+mi}{20}\PY{p}{,}\PY{l+m+mi}{2}\PY{p}{)}
        \PY{c+c1}{\PYZsh{} our infrared filters are pretty lousy ...}
        \PY{n}{delta} \PY{o}{=} \PY{n}{nd}\PY{o}{.}\PY{n}{random\PYZus{}normal}\PY{p}{(}\PY{n}{shape}\PY{o}{=}\PY{p}{(}\PY{n+nb}{len}\PY{p}{(}\PY{n}{wavelengths}\PY{p}{)}\PY{p}{)}\PY{p}{)} \PY{o}{*} \PY{l+m+mi}{1}
        
        \PY{n}{radiance} \PY{o}{=} \PY{n}{wien}\PY{p}{(}\PY{n}{wavelengths} \PY{o}{+} \PY{n}{delta}\PY{p}{,}\PY{n}{realtemp}\PY{p}{)}
        \PY{n}{plt}\PY{o}{.}\PY{n}{plot}\PY{p}{(}\PY{n}{wavelengths}\PY{o}{.}\PY{n}{asnumpy}\PY{p}{(}\PY{p}{)}\PY{p}{,} \PY{n}{radiance}\PY{o}{.}\PY{n}{asnumpy}\PY{p}{(}\PY{p}{)}\PY{p}{,} \PY{n}{label}\PY{o}{=}\PY{l+s+s1}{\PYZsq{}}\PY{l+s+s1}{measured}\PY{l+s+s1}{\PYZsq{}}\PY{p}{)}
        \PY{n}{plt}\PY{o}{.}\PY{n}{plot}\PY{p}{(}\PY{n}{wavelengths}\PY{o}{.}\PY{n}{asnumpy}\PY{p}{(}\PY{p}{)}\PY{p}{,} \PY{n}{wien}\PY{p}{(}\PY{n}{wavelengths}\PY{p}{,} \PY{n}{realtemp}\PY{p}{)}\PY{o}{.}\PY{n}{asnumpy}\PY{p}{(}\PY{p}{)}\PY{p}{,} \PY{n}{label}\PY{o}{=}\PY{l+s+s1}{\PYZsq{}}\PY{l+s+s1}{true}\PY{l+s+s1}{\PYZsq{}}\PY{p}{)}
        \PY{n}{plt}\PY{o}{.}\PY{n}{legend}\PY{p}{(}\PY{p}{)}
        \PY{n}{plt}\PY{o}{.}\PY{n}{show}\PY{p}{(}\PY{p}{)}
\end{Verbatim}


    \begin{center}
    \adjustimage{max size={0.9\linewidth}{0.9\paperheight}}{output_13_0.png}
    \end{center}
    { \hspace*{\fill} \\}
    
    Use Gluon to estimate the real temperature based on the variables
\texttt{wavelengths} and \texttt{radiance}.

\begin{itemize}
\tightlist
\item
  You can use Wien's law implementation \texttt{wien(lam,t)} as your
  forward model.
\item
  Use the loss function \(l(y,y') = (\log y - \log y')^2\) to measure
  accuracy.
\end{itemize}


    % Add a bibliography block to the postdoc
    
    
    
    \end{document}
